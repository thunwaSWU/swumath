%%%%%%%%%%%%%%%%%%%%%%%%%%%%%%%%%%%%%%%%%%%
%% This is a sample tex file for swumath-thai.cls
%% Copyright (C) 2020- Thunwa Theerakarn
%% Permission is hereby granted, free of charge, to any person obtaining a copy of this software 
%% and associated documentation files (the "Software"),to deal in the Software without restriction, 
%% including without limitation the rights to use, copy, modify, merge, publish, distribute, sublicense, 
%% and/or sell copies of the Software, and to permit persons to whom the Software is furnished to 
%% do so, subject to the following conditions:
%%
%% The above copyright notice and this permission notice shall be included in all copies or substantial portions of the Software.
%%
%% THE SOFTWARE IS PROVIDED "AS IS", WITHOUT WARRANTY OF ANY KIND, 
%% EXPRESS OR IMPLIED, INCLUDING BUT NOT LIMITED TO THE WARRANTIES OF 
%% MERCHANTABILITY, FITNESS FOR A PARTICULAR PURPOSE AND NONINFRINGEMENT. 
%% IN NO EVENT SHALL THE AUTHORS OR COPYRIGHT HOLDERS BE LIABLE FOR ANY 
%% CLAIM, DAMAGES OR OTHER LIABILITY, WHETHER IN AN ACTION OF CONTRACT, 
%% TORT OR OTHERWISE, ARISING FROM, OUT OF OR IN CONNECTION WITH THE 
%% SOFTWARE OR THE USE OR OTHER DEALINGS IN THE SOFTWARE.
%%
%%
%% You must have swumath.cls, swumath-thai.cls, and Srinakharinwirot_Logo_TH_Color.png, 
%% and all 4 TH Sarabun New font files in the same folder.
%%
%% Options for swumath class : ma491, ma493, st472, or st474
%%
%% Preloaded Packages : geometry, graphicx, enumitem
%%                                     amssymb, amsmath, amsthm
%% For basic usage, you do not need to load any other packages or redefine anything.
%%%%%%%%%%%%%%%%%%%%%%%%%%%%%%%%%%%%%%%%%%%
%%%%%%%%%%%%%%%%%%%%%%%%%%%%%%%%%%%%%%%%%%%
%% NOTICE
%%
%% This sample is for compiling on your local system.
%% For Overleaf, use demo_thai_overleaf.tex
%%
%% 1. MUST COMPILE WITH XeLaTeX
%% 2. USE UTF-8 for source
%%
%%%%%%%%%%%%%%%%%%%%%%%%%%%%%%%%%%%%%%%%%%%

\documentclass[ma493]{swumath-thai}

%%%%%%%%%%%%%%%%%%%%%%%
% SETUP FOR THAI LANGUAGE
%%%%%%%%%%%%%%%%%%%%%%%

\usepackage{fontspec}
\usepackage{xltxtra}
\usepackage{xunicode}
\XeTeXlinebreaklocale "th"
\XeTeXlinebreakskip = 0pt plus 0pt 
\usepackage{fonts-tlwg}
\defaultfontfeatures{Mapping=tex-text} 

%\newfontfamily{\thaifont}{THSarabunNew}[Scale = 1.3]

\newfontfamily{\thaifont}{THSarabunNew.ttf}[
	Path, 
	BoldFont 	= THSarabunNew Bold.ttf,
       	ItalicFont  = THSarabunNew Italic.ttf,
       	BoldItalicFont = THSarabunNew BoldItalic.ttf,
	Script = Thai,
	Scale = 1.3]

%% Ucharclasses Package is responsible for auto-switching font based on Unicode
\usepackage[Latin,Thai]{ucharclasses}
\setTransitionsForLatin
  {\begingroup\rmfamilylatin}
  {\endgroup}
  
%% Polyglossia
\usepackage{polyglossia}
\setmainlanguage{thai}

%\usepackage[thai,english]{babel}

%Uncomment this for Overleaf
%\setmainfont[
%  BoldFont={THSarabunNew Bold.ttf},
%  ItalicFont={THSarabunNew Italic.ttf},
%  BoldItalicFont={THSarabunNew BoldItalic.ttf},
%]{THSarabunNew.ttf}

%Thai names for environments defined in class
\addto\captionsthai{%
  \renewcommand{\figurename}{รูป}%
  \renewcommand{\bibname}{เอกสารอ้างอิง}
  \renewcommand*{\proofname}{บทพิสูจน์}
  \renewcommand{\listfigurename}{สารบัญภาพ}
}

%%%%%%%%%%%%%%%%%%%%%%%

%%%%%%%%%%%%%%%%%%%%%%%
%% Parameters
%% Most are self-explanatory. 
%%%%%%%%%%%%%%%%%%%%%%%

\academicyear{2563}

%Project title
\title{จัตุรัสวิเศษในสมัยกรุงศรีอยุธยา}
\engtitle{Magic Squares in the Ayutthaya Era}

%Students (up to 4)
\numberofmembers{3}

\nameone{เฮอร์ไมโอนี}
\lastnameone{เกรนเจอร์}
\idone{6X102010XX1}

\nametwo{แฮรรี่}
\lastnametwo{พอตเตอร์}
\idtwo{6X102010XX2}

\namethree{รอน}
\lastnamethree{วีสลีย์}
\idthree{6X102010XX3}

%\namefour{Ron}
%\lastnamefour{Weasley}
%\idfour{6X102010XX3}

%Do not include Ph.D. for advisor or co-advisor
\advisor{ศาสตราจารย์ ดร.อัลบัส ดัมเบิลดอร์}
%\coadvisor{YYYY}

%Committee
\numberofexaminers{3}

%Uncomment as you need
%\examone{Examiner 1}
%\examtwo{Examiner 2}
%\examthree{Examiner 3}
%\examfour{Examiner 3}

%%%%%%%%%%%%%%%%%%%%%%%
%% Your document starts here
%%%%%%%%%%%%%%%%%%%%%%%

\begin{document}

%%% Front Matter
\frontmatter

\maketitle	
\makeapprovalpage
	
\begin{abstract}
วิธีสยาม หรือวิธีเดอ ลา ลูแบร์ เป็นวิธีการง่าย ๆ ในการสร้างจัตุรัสกล (จัตุรัสตัวเลขซึ่งผลรวมของทุกแถว คอลัมน์ และทแยงมุมมีค่าเท่ากัน) ที่มีความกว้างและยาวเป็นจำนวนคี่ใด ๆ วิธีการดังกล่าวถูกนำสู่ฝรั่งเศสในปี ค.ศ. 1688 โดยนักคณิตศาสตร์และทูตชาวฝรั่งเศส ซีมง เดอ ลา ลูแบร์ เมื่อเขาเดินทางกลับประเทศหลังการเดินทางมาเป็นคณะทูตที่ราชอาณาจักรสยามเมื่อปี ค.ศ. 1687 วิธีสยามทำให้การสร้างจัตุรัสกลเป็นไปอย่างตรงไปตรงมา
\end{abstract}

%\begin{acknowledgment}
%	Thanks.
%\end{acknowledgment}

\tableofcontents
%\listoftables
%\clearpage
%\listoffigures

%%% Main Matter
\mainmatter

\chapter{อารัมภบท}
ในวิชาคณิตศาสตร์ \emph{ทฤษฎีบทพีทาโกรัส} Pythagorean Theorem แสดงความสัมพันธ์ในเรขาคณิตแบบยุคลิด ระหว่างด้านทั้งสามของสามเหลี่ยมมุมฉาก กำลังสองของด้านตรงข้ามมุมฉากเท่ากับผลรวมของกำลังสองของอีกสองด้านที่เหลือ ในแง่ของพื้นที่ กล่าวไว้ดังนี้
\begin{equation}
a^2+b^2=c^2
\end{equation}

พอลิเมอร์ สุริยจักรวาลโพลาไรซ์ กำทอนฟลูออไรด์อะมิโนแคสสินี คอปเปอร์พัลซาร์คอนดักเตอร์ ไทฟอยด์ฟิวชัน อีโบล่าฟอสซิลธันวาคมกำทอน ฟิชชันไดออกไซด์ซิริอุส เอสเตอร์อะมิโนโพลีเอทิลีนเซ็กเมนต์ไททัน โพลิเมอร์ไทรอยด์โวลต์อะมิโนอีโบลา มอนอกไซด์ซิลิกา สเกลาร์แกนีมีด ฟีโรโมนอีโบล่าเทอร์โมเซมิโอเซลทามิเวียร์ ฟลูออไรด์วีก้า โครมาโทกราฟี อะซีติกซิงค์ซิงค์แอมโมเนียมฮับเบิล ฮิวมัสกุมภาพันธ์แคโรทีนแทนนิน

ซิลิกาอินทิกรัลเวสิเคิลแอลกอฮอลิซึม ออโรร่าฮิวมัสไททัน ฟลาโวนอยด์ไททัน เมทริกซ์เวก้า ไดนามิกแอสพาร์แตมสเปิร์ม จุลชีววิทยากำทอนเซ็กเตอร์ แคโรทีนพาราโบลาเวก้าไฮโดรลิก ดอปเปลอร์ฟลาโวนอยด์ ฟิชชันพันธุศาสตร์ซัลเฟตเคอราติน ซิลิกา พันธุศาสตร์ฟิชชัน ปฏิยานุพันธ์อะซีติกออโรราไทฟอยด์ซิลิกา ไพรเมต สเปิร์มแคโรทีนอัลคาไลน์มกราคมไททัน มกราคมคูลอมบ์พฤษภาคมกำทอนไฮเพอร์โบลา เพอร์ออกไซด์สเกลาร์เมทริกซ์

\begin{figure}[h]
\begin{center}
\includegraphics[width=2in]{Srinakharinwirot_Logo_EN_Color.png}
\caption{test caption}
\end{center}
\end{figure}

\chapter{ความรู้พื้นฐาน}
ในบทนี้เราจะพูดถึงนิยามและทฤษฎีบทที่จำเป็น
 
\section{นิยามพื้นฐาน} 

\begin{definition}
\emph{อินทิกรัลสามชั้น} (triple integral) ของ $f$ บนกล่องสี่เหลี่ยมมุมฉาก $B$ คือ
\[ \iiint_B f(x,y,z) \, dV = \lim_{L, M, N \to \infty} \sum_{i = 1}^L \sum_{j = 1}^M \sum_{k = 1}^N f(x_{ijk}^*, y_{ijk}^*, z_{ijk}^*) \Delta V \]
เมื่อลิมิตหาค่าได้
\end{definition}
 
\section{ทฤษฎีบทที่จำเป็น}

ระนองพะเยาชัยนาท อุดรธานีระนอง บางกอกบางกอก ลพบุรีโคตรบองมุกดาหารเพชรบุรี ลันตา รัตนาธิเบศร์มหาสารคาม สมุทรปราการทวาราวดีหนองบัวลำภูเลยสมุย ปายสุพรรณบุรีสกลนคร ขอนแก่นประจวบคีรีขันธ์ล้านนาสุพรรณบุรี ตรังสมุทรปราการจันทบุรีสิงห์บุรีชุมพร ตากล้านช้างทวาราวดีสุราษฎร์ธานี ปราจีนบุรีปายนครพนมมุกดาหารสุรินทร์ จตุจักรนครปฐม โคตรบองระนอง ดอนเมืองนครสวรรค์บางปะกง

\begin{theorem}[ทฤษฎีบทของฟูบินี]
ให้ $f$ เป็นฟังก์ชันต่อเนื่องบนสี่เหลี่ยม $R = [a,b] \times [c,d]$ จะได้ว่า
\[ \iint_R f(x,y) dA = \int_a^b \int_c^d f(x,y) \, dy \, dx = \int_c^d \int_a^b f(x,y) \, dx \, dy \]
\end{theorem}
\begin{proof}
เป็นการบ้านสำหรับผู้อ่าน
\end{proof}

\chapter{เลขวิทยุของกราฟวัฏจักร}

\section{เลขวิทยุ}

\begin{lemma}
เล็มมา
\end{lemma}

\begin{proposition}[กฎของลิมิต]
ให้ $f$ และ $g$ เป็นฟังก์ชันสองตัวแปรที่มีโดเมนคือ $D \subset \mathbb{R}^2$ 
ให้ $(a,b)$ เป็นจุดภายในของ $D$ และให้ $k$ เป็นค่าคงที่ 

ถ้า $\displaystyle \lim_{(x,y) \to (a,b)} f(x,y) = L$ และ 
$\displaystyle \lim_{(x,y) \to (a,b)} g(x,y) = M$ จะได้ว่า
\begin{enumerate}
\item $\lim_{(x,y) \to (a,b)} x = a$ และ $\lim_{(x,y) \to (a,b)} y = b$
\item $\lim_{(x,y) \to (a,b)} k = k$
\item $\lim_{(x,y) \to (a,b)} f(x,y) + g(x,y) = L + M$
\item $\lim_{(x,y) \to (a,b)} f(x,y)g(x,y) = LM$
\item $\lim_{(x,y) \to (a,b)} \dfrac{f(x,y)}{g(x,y)} = \dfrac{L}{M}$ เมื่อ $M \neq 0$
\item $\lim_{(x,y) \to (a,b)} |f(x,y)| = |L|$
\item $\lim_{(x,y) \to (a,b)} (f(x,y))^k = L^k$ เมื่อ $L^k$ มีความหมาย 
\end{enumerate}
\end{proposition}
\begin{proof}
	เป็นการบ้านของผู้อ่าน
\end{proof}

%%%%%%%%%%%%%%%%%%%%%%
%% APPENDIX
%%%%%%%%%%%%%%%%%%%%%%

\appendix
%Change appendix counter to Thai alphabet
\renewcommand{\thechapter}{\thaiAlph{chapter}} 

\chapter{แบบฝึกหัดสำหรับผู้อ่าน}

\begin{itemize}
\item สระน้ำโบราณสระหนึ่งเป็นรูปสี่เหลี่ยมผืนผ้ากว้าง 15 เมตร ยาว 25 เมตร นักโบราณคดีวัดความสูงของน้ำในสระทุก ๆ 5 เมตร โดยเริ่มจากมุมหนึ่งของบ่อ ตารางข้างล่างแสดงความสูงของน้ำในสระที่วัดได้ในแต่ละจุด จงหาค่าประมาณของปริมาตรของน้ำในสระ
\begin{table}[h]
\begin{center}
\begin{tabular}{|c|c|c|c|c|c|c|}
\hline
& 0 & 5 & 10 & 15 & 20 & 25 \\ \hline
0 & 2 & 3 & 5 & 6 & 8 & 7 \\ \hline
5 & 2 & 3 & 5 & 8 & 8 & 7 \\ \hline
10 & 2 & 3 & 7 & 9 & 10 & 8 \\ \hline
15 & 3 & 2 & 5 & 8 & 8 & 6 \\ \hline
20 & 2 & 2 & 5 & 6 & 8 & 6 \\ \hline
\end{tabular}
\caption{ความสูงของน้ำในสระ}
\end{center}
\end{table}

\item จงหาค่าของอินทิกรัลต่อไปนี้โดยมองว่าเป็นปริมาตรของทรงตัน
\begin{enumerate}
\item $\displaystyle \iint_R \sqrt{3} \, dA$ เมื่อ $R = \{(x,y) : 1 \leq x \leq 5, -1 \leq y \leq 4\}$
\item $\displaystyle \iint_R 2x+1 \, dA$ เมื่อ $R = \{(x,y) : 0 \leq x \leq 2, 0 \leq y \leq 4\}$
\end{enumerate}

\item จงหาค่าของอินทิกรัลต่อไปนี้
\begin{enumerate}
\item $\displaystyle \int_0^1 \int_1^2 (4x^3 - 9x^2y^2) \, dy \, dx$
\item $\displaystyle\int_0^1 \int_0^3 e^{x+3y} \, dx \, dy$
\item $\displaystyle\int_0^1 \int_0^1 \dfrac{x}{1+xy} \, dx \, dy$
\end{enumerate}

\end{itemize}

\newpage

%You can use the following line to manually set page number.
%\setcounter{page}{20}

\begin{thebibliography}{99}
	
\bibitem{davis} B. Davis, D. Maclagan, The card game SET, \emph{Math. Intelligencer}, 25 no. 3 (2003), 33-40, http://dx.doi.org/10.1007/bf02984846.
\bibitem{goldberg} T. Goldberg. Algebra From Geometry in the Card Game SET. \emph{The College Mathematics Journal}, 47 no.4 (2016), 265-273.
\bibitem{gordon} H. Gordon, R. Gordon, E. McMahon, Hands-on SET, \emph{PRIMUS}, 23 (2013) 646-658.  
\bibitem{lang} S. Lang, \emph{Linear Algebra Third Edition}. New haven, USA, 1987.
	
\end{thebibliography}

\end{document}
